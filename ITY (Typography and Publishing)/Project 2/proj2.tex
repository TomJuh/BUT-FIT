\documentclass[a4paper, twocolumn,11pt]{article}
\usepackage[utf8]{inputenc}
\usepackage[IL2]{fontenc}
\usepackage{times}
\usepackage[czech]{babel}
\usepackage{mathptmx}
\usepackage{amsthm}
\usepackage{amsmath}
\usepackage{amssymb}
\theoremstyle{definition}
\newtheorem{definition}{Definice}
\newtheorem{sentence}{Věta}
\usepackage{geometry}
    \geometry{
    a4paper,
    total={18cm,25cm},
    left=1.5cm,
    top=2.5cm,
    }
\begin{document}

\begin{titlepage}

\begin{center}
\Huge
\textsc{Fakulta informačních technologií
Vysoké učení technické v Brně} \\
\vspace{\stretch{0.3}}

\huge
Typografie a publikování -- 2. projekt \\
Sazba dokumentů a matematických výrazů \\
\vspace{\stretch{0.4}}
\end{center}

{\LARGE 22.3.2021 \hfill
Tomáš Juhász (xjuhas04)}
\end{titlepage}
\section*{Úvod}
V této úloze si vyzkoušíme sazbu titulní strany, matematických vzorců, prostředí a dalších textových struktur obvyklých pro technicky zaměřené texty (například rovnice (1) nebo Definice 1 na straně 1). Rovněž si vyzkoušíme používání odkazů \verb!\ref! a \verb!\pageref!. 

Na titulní straně je využito sázení nadpisu podle optického středu s využitím zlatého řezu. Tento postup byl probírán na přednášce. Dále je použito odřádkování se zadanou relativní velikostí 0.4 em a 0.3 em. 

V případě, že budete potřebovat vyjádřit matematickou konstrukci nebo symbol a nebude se Vám dařit jej nalézt v samotném \LaTeX u, doporučuji prostudovat možnosti balíku maker \AmS-\LaTeX

\section{Matematický text}
Nejprve se podíváme na sázení matematických symbolůa výrazů v plynulém textu včetně sazby definic a vět s vy-užitím balíkuamsthm. Rovněž použijeme poznámku podčarou s použitím příkazu \verb!footnote!. Někdy je vhodnépoužít konstrukci \verb!\mvox{}!, která říká, že text nemá být zalomen.

\begin{definition}\label{def1} \emph{Rozšířený zásobníkový automat} (RZA) je de- finován jako sedmice tvaru $A = (Q,\Sigma,\Gamma,\Delta, q_0 , Z_0 , F ), Q$ kde:
\begin{itemize}
    \item $Q$ \emph{je konečná množina} vnitřních (řídicích) stavů,
    \item $\Sigma$ \emph{je konečná} vstupní abeceda,
    \item $\Gamma$ \emph{je konečná} zásobníková abeceda,
    \item $\delta$ \emph{je} přechodová funkce $Q\times(\Sigma\cup\{\epsilon\})\times\Gamma^* \rightarrow 2^{Q\times\Gamma^*}$,
    \item $q_0 \in Q je$ počáteční stav,$Z_0 \in \Gamma je$ startovací symbol zásobníku \emph{a} $F \subseteq$ \emph{je množina} koncových stavů.
\end{itemize}

Nechť $P = (Q,\Sigma,\Gamma,\Delta, q_0 , Z_0 , F)$ je rozšířený zásob-níkový automat. \emph{Konfigurací} nazveme trojici $(q,w,a) \in Q \times \Sigma^*\times \Gamma^*$,~ kde~  $q$ je aktuální stav vnitřního řízení,
$w$ je dosud nezpracovaná část vstupního řetězce a $a = Z_{i1}Z_{i2}...Z_{ik}$ je obsah zásobníku\footnote{$Z_{i1}$ je vrchol zásobníku}.
\end{definition}
\subsection{Podsekce obsahující větu a odkaz}

\begin{definition}\label{def2}
 Řetězec $w$ nad abecedou $\Sigma$ je přijat RZA\\ \emph{A jestliže} $(q_0,w,Z_0)$~ 
$\overset{*}{\underset{A}{\vdash}}$~  $(q_F,\epsilon,\gamma)$ \emph{pro nějaké} $\gamma \in \Gamma^*$ \emph{a} $q_F \in F$.~ \emph{Množinu}~ \emph{L(A)}~ $=$~ \{\emph{w~ $|$~ w je přijat RZA A}\}~ $\subseteq$~ $\Sigma^*$
\emph{nazýváme} jazyk přijímaný RZA \emph{A}.


\end{definition}

Nyní si vyzkoušíme sazbu vět a důkazů opět s použitímbalíku \verb!amsthm!.
\begin{sentence}
\emph{Třída jazyků, které jsou přijímány ZA, odpovídá} bezkontextovým jazykům.
\begin{proof}
V důkaze vyjdeme z Definice 1 a 2. \end{proof}

\end{sentence}

\section{Rovnice a odkazy}
Složitější matematické formulace sázíme mimo plynulýtext. Lze umístit několik výrazů na jeden řádek, ale pak jetřeba tyto vhodně oddělit, například příkazem \verb!\quad!.


$$
\begin{aligned}
\sqrt[i]{x_{i}^{3}} \quad \text{kde $x_i$ je $i$-té sudé číslo splňující} \quad x_{i}^{x_{i}^{i^{2}}+2} \leq y_{i}^{x_{i}^{4}}
\end{aligned}
$$

V rovnici (1) jsou využity tři typy závorek s různouexplicitně definovanou velikostí.

\begin{align}
x&=\left[\{[a+b] * c\}^{d} \oplus 2\right]^{3 / 2}
\\
y&=\lim _{x \rightarrow \infty} \frac{\frac{1}{\log _{10} x}}{\sin ^{2} x+\cos ^{2} x}
\nonumber
\end{align}


V této větě vidíme, jak vypadá implicitní vysázení li-mitylim $n \rightarrow \infty f(n)$ v normálním odstavci textu. Podobněje to i s dalšími symboly jako $\prod_{i=1}^{n} 2^{i}$ či $\bigcap_{A \in B} A$.V pří-padě vzorců $\lim\limits_{n\to\infty}f(n)$
 a $\prod\limits_{i=1}^{n} 2^{i}$
 jsme si vynutili méněúspornou sazbu příkazem \verb!\limits!
\begin{equation}
\int_{b}^{a} g(x) \mathrm{d} x=-\int_{a}^{b} f(x) \mathrm{d} x
\end{equation}
\section{Matice}
Pro sázení matic se velmi často používá prostředí \verb!arraya! závorky (\verb!\left!,\verb!\right!).

$$\left(\begin{array}{ccc}
a-b & \widehat{\xi+\omega} & \pi \\
\overrightarrow{\mathbf{a}} & \overleftrightarrow{A C} & \hat{\beta}
\end{array}\right)=1 \Longleftrightarrow \mathcal{Q}=\mathbb{R}$$

$$\mathbf{A}=\left\|\begin{array}{cccc}
a_{11} & a_{12} & \ldots & a_{1 n} \\
a_{21} & a_{22} & \ldots & a_{2 n} \\
\vdots & \vdots & \ddots & \vdots \\
a_{m 1} & a_{m 2} & \ldots & a_{m n}
\end{array}\right\|=\left|\begin{array}{cc}
t & u \\
v & w
\end{array}\right|=t w-u v$$

Prostředí \verb!array! lze úspěšně využít i jinde.
$$\left(\begin{array}{l}
n \\
k
\end{array}\right)=\left\{\begin{array}{cl}
0 & \text { pro } k<0 \text { nebo } k>n \\
\frac{n !}{k !(n-k) !} & \text { pro } 0 \leq k \leq n.
\end{array}\right.$$

\end{document}
